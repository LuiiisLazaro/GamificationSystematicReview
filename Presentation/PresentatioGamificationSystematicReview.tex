%%%%%%%%%%%%%%%%%%%%%%%%%%%%%%%%%%%%%%%%%
% Beamer Presentation
% LaTeX Template
% Version 1.0 (10/11/12)
%
% This template has been downloaded from:
% http://www.LaTeXTemplates.com
%
% License:
% CC BY-NC-SA 3.0 (http://creativecommons.org/licenses/by-nc-sa/3.0/)
%
%%%%%%%%%%%%%%%%%%%%%%%%%%%%%%%%%%%%%%%%%

%----------------------------------------------------------------------------------------
%	PACKAGES AND THEMES
%----------------------------------------------------------------------------------------

\documentclass{beamer}

\mode<presentation> {
	\usetheme{CambridgeUS}
	\usecolortheme{seahorse}
	\setbeamertemplate{navigation symbols}{}
}
\usepackage[spanish]{babel}
\usepackage[utf8]{inputenc}
\usepackage{graphicx}
\usepackage{booktabs}
\usepackage[spanish]{babel}
\usepackage{comment}
\usepackage{multirow}
%----------------------------------------------------------------------------------------
%	TITLE PAGE
%----------------------------------------------------------------------------------------

\title[Gamificación]{Métodos y Técnicas de Gamificación Implementadas en Educación: Revisión Sistemática\\ Presentación de Avances} %

\author{Luis Angel Hernández Lázaro} % 
\institute[CIMAT] % 
{
Centro de Investigación en Matemáticas A.C. Unidad Zacatecas \\ % 
\medskip
\textit{luis.hernandez@cimat.com} % 
}
\date{\today} % 

\begin{document}

%----------------------------------------------------------------------------------------
%	PRESENTATION
%----------------------------------------------------------------------------------------
\begin{frame}
\titlepage

\end{frame}

\begin{frame}
\frametitle{Agenda}

\begin{columns}[c] % The "c" option specifies centered vertical alignment while the "t" option is used for top vertical alignment
	
	\column{.45\textwidth} % Left column and width
	\begin{figure}
		\begin{center}
			\includegraphics[scale=0.25]{images/2icons/content.png}
		\end{center}
	\end{figure}
	
	\column{.5\textwidth} % Right column and width
	\tableofcontents 
	
\end{columns}
\end{frame}

%----------------------------------------------------------------------------------------
%	PRESENTATION SLIDES
%----------------------------------------------------------------------------------------

%------------------------------------------------
\section{Introducción} %
%------------------------------------------------

\begin{frame}
\Huge{\centerline{Introducción}}
\begin{figure}
	\begin{center}
		\includegraphics[scale=0.2]{images/2icons/start.png}
	\end{center}
\end{figure}
\end{frame}

\begin{frame}
	\frametitle{Motivación}
	Debido al interés en el de conocer acerca del tema, en especial la gamificación aplicada en la educación.	\\~\\
	Saber como la gamificación puede mejorar la motivación para el aprendizaje de los estudiantes y mejorar su desempeño académico.	\\~\\
	Qué métodos o técnicas existen para ser aplicadas en la educación.
	\\~\\
	
	\begin{figure}
		\begin{center}
			\includegraphics[scale=0.1]{images/2icons/student.png}
		\end{center}
	\end{figure}
\end{frame}

\begin{frame}
	\frametitle{BackGround}
	\begin{description}
		\item[Gamificación]: Gamification es la aplicación de la mecánica de juego y el jugador, incentivado a los entornos que no son juegos (O'Donovan, gain, \& Marais,2013).
		\item[Elementos]: para el salón de clases
		\begin{itemize}
			\item Retroalimentación
			\item Colaboración
			\item Puntuaciones, Niveles, Premios
			\item Quest (Actividades, Tareas, Trabajos)
			\item Historias
			\item Mapa de Conocimiento.
		\end{itemize} 
		(Villagrasa \& Duran, 2013)
	\end{description}
\end{frame}

\section{Revisión Sistemática}
\begin{frame}
\Huge{\centerline{Revisión Sistemática}}
	\begin{figure}
		\begin{center}
			\includegraphics[scale=0.4]{images/2icons/systematicReview.png}
		\end{center}
	\end{figure}
\end{frame}

\subsection{Proceso}
\begin{frame}
	\frametitle{El Proceso de la Revisión Sistemática}	
	
	\begin{table}
		\begin{center}
			\begin{tabular}{| p{5cm} | l |}
				\hline 
				\multicolumn{1}{|c|}{\textbf{Etapa}} & \multicolumn{1}{|c|}{\textbf{Paso}}\\
				\hline
				\multicolumn{1}{|c|}
				{\multirow{4}{*}
					{\begin{tabular}[c]{@{}c@{}}
							Planificación de\\ la Revisión
						\end{tabular}}
					} & Identificación de la necesidad para realizar la revisión.\\ \cline{2-2} 
					\multicolumn{1}{|c|}{} & Especificación de la(s) pregunta(s) de investigación.\\ \cline{2-2} 
					\multicolumn{1}{|c|}{} & Desarrollo del protocolo de revisión.\\ \cline{2-2} 
					\multicolumn{1}{|c|}{} & Evaluación del protocolo de revisión.\\ 
					\hline
					\multicolumn{1}{|c|}
					{\multirow{5}{*}
						{\begin{tabular}[c]{@{}c@{}}
								Ejecución de\\ la Revisión
							\end{tabular}}
						} & Identificación de la investigación.\\ \cline{2-2} 
						\multicolumn{1}{|c|}{} & Selección de estudios primarios.\\ \cline{2-2} 
						\multicolumn{1}{|c|}{} & Evaluación de la calidad de los estudios.\\ \cline{2-2} 
						\multicolumn{1}{|c|}{} & Extracción y monitoreo de datos.\\ \cline{2-2} 
						\multicolumn{1}{|c|}{} & Sintetizar los datos.\\ 
						\hline
						\multicolumn{1}{|c|}
						{\multirow{3}{*}
							{\begin{tabular}[c]{@{}c@{}}
									Reportar\\ Resultados
								\end{tabular}}
							} & Especificación de los mecanismos de difusión.\\ \cline{2-2} 
							\multicolumn{1}{|c|}{} & Formateo del informe principal.\\ \cline{2-2} 
							\multicolumn{1}{|c|}{} & Evaluación del reporte.\\ 
							\hline
				\end{tabular}
		\end{center}
	\end{table}	
\end{frame}


%------------------------------------------------
%------------------------------------------------
\subsection{Planeación de la Revisión}
\begin{frame}
\Huge{\centerline{Planeación de la Revisión}}
	\begin{figure}
		\begin{center}
			\includegraphics[scale=.4]{images/2icons/plan.png}
		\end{center}
	\end{figure}
\end{frame}

\begin{frame}
    \frametitle{Identificación de la Necesidad de la Revisión Sistemática}
    Hoy en día la Gamificación es un tema que esta siendo aplicado más allá de los juegos que tradicionalmente conocemos, la aplicación de técnicas para gamificación pueden ser variadas dependiendo del área donde se trabaje. \\
    La necesidad surge para conocer si las técnicas o métodos de Gamificación desarrollados pueden ser aplicados en la educación para mejorar el aprendizaje de los estudiantes.
    	\begin{figure}
    		\begin{center}
    			\includegraphics[scale=0.1]{images/2icons/guide.png}
    		\end{center}
    	\end{figure}
\end{frame}

\begin{frame}
	\frametitle{Objetivos: General y Específicos (1/2)}
	Como parte de la definición de nuestro Objetivo General del estudio tenemos:
	\begin{description}
		\item[Definir] el estado del arte actual del uso de técnicas de Gamificación en la educación tradicional (Aula, Alumno, Profesor), para descubrir las diferentes técnicas de Gamificación implementadas en la educación.
	\end{description}
	\begin{figure}
		\begin{center}
			\includegraphics[scale=0.1]{images/2icons/tarjet.png}
		\end{center}
	\end{figure}
\end{frame}

\begin{frame}
	\frametitle{Objetivos: General y Específicos(2/2)}
	Los objetivos Específicos del estudio son:
	\begin{description}
		\item[Definir] el estado del arte de la Gamificación en la educación.
		\item[Revisar] las estrategias o técnicas de Gamificación implementadas en el área de la educación.
		\item[Comparar] la aplicación de técnicas de gamificación en la educación.
	\end{description}
	\begin{figure}
		\begin{center}
			\includegraphics[scale=0.1]{images/2icons/tarjet2.png}
		\end{center}
	\end{figure}
\end{frame}


\begin{frame}
    \frametitle{Especificación de las Preguntas de Investigación}
    \begin{table}
                \begin{center}
                    \label{table:researchQuestions}
                    \begin{tabular}{| p{5.5cm} | p{5.5cm} |}
                        \hline
                        \multicolumn{1}{|c|}{\textbf{Pregunta}}  & \multicolumn{1}{|c|}{\textbf{Objetivo}} \\
                        \hline
                        ¿Qué elementos de gamificación son usados en la educación? & Identificar los elementos de Gamificación existentes, en particular en la educación. \\
                        \hline
                        ¿Qué tipo de estudios de investigación aplican gamificación en la educación? & Descubrir los trabajos que han sido desarrollados para aplicar la gamificación en la educación. \\
                        \hline
                        ¿Qué técnicas de gamificación han sido más efectivas? & De las técnicas existentes cuales han tenido resultados positivos en la educación y porque. \\
                        \hline
                        ¿La gamificación promueve el aprendizaje? & Verificar si el conocimiento de los estudiantes mejora con la aplicación de gamificación.\\
                        \hline
                        ¿En qué nivel educativo ha sido investigada la gamificación? & Conocer que experiencias se han tenido en el uso de gamificación.\\
                        \hline
                    \end{tabular}
                \end{center}
            \end{table}
\end{frame}

\begin{comment}
\begin{frame}
    \frametitle{Evaluación del Protocolo de Revisión}
    Como parte de la evaluación del protocolo se realizaron revisiones con la Dra. Mirna profesora de la materia de Estudio Guiado I y el Dr. Arturo asesor para el realizar este trabajo.\\
    Observaciones para evaluar el trabajo:
    \begin{itemize}
    	\item Actualización de la palabra clave  (KeyWord) ``Serious Game" por ``Serious Games".
    \end{itemize}
    \begin{figure}
    	\begin{center}
    		\includegraphics[scale=.1]{images/2icons/checklist.png}
    	\end{center}
    \end{figure}
    
\end{frame}
\end{comment}
%------------------------------------------------
%------------------------------------------------
\subsection{Ejecución de la Revisión} %
\begin{frame}
\Huge{\centerline{Ejecución de la Revisión}}
	\begin{figure}[H]
		\begin{center}
		    \includegraphics[scale=.3]{images/2icons/execute.png}
	    \end{center}
	\end{figure}
\end{frame}

\begin{frame}
	\frametitle{Selección de las Fuentes de Investigación (1/3)\\Palabras Clave}
	\begin{columns}[c] % The "c" option specifies centered vertical alignment while the "t" option is used for top vertical alignment
		\column{.5\textwidth} % Left column and width
		Definición de Palabras Clave:
		\begin{itemize}
			\item Gamification
			\item Gamify
			\item Serious Games
			\item Education
			\item E-Learning
		\end{itemize}
		\column{.3\textwidth} % Right column and width
		\begin{figure}
			\begin{center}
				\includegraphics[scale=0.2]{images/2icons/keyword.png}
			\end{center}
		\end{figure}	    	
	\end{columns}
\end{frame}


\begin{frame}
	\frametitle{Selección de las Fuentes de Investigación (2/3)\\ Cadenas de Búsqueda}
	\begin{table}
		\begin{center}
			\begin{tabular}{| l | p{10cm} |}
				\hline
				\multicolumn{1}{|c|}{\textbf{Núm}} & \multicolumn{1}{|c|}{\textbf{Cadena de Búsqueda}} \\
				\hline
				SS1 & ( ``Gamification''{ }AND (``Education''{ }OR ``E-Learning''{ }) )\\
				\hline
				SS2 & ( ``Serious Games''{ }AND ``Gamification''{ }AND (``Education''{ }OR ``E-Learning''{ }) )\\
				\hline            
				SS3 & ( ``Gamify''{ }AND (``Education''{ }OR ``E-Learning''{ }) )\\
				\hline
			\end{tabular}
		\end{center}
	\end{table}
	    \begin{columns}[c]
	    	\column{.2\textwidth}
	    	\begin{figure}
	    		\begin{center}
	    			\includegraphics[scale=.5]{images/2icons/r.png}
	    		\end{center}
	    	\end{figure}
	    	\column{.2\textwidth}
	    	\begin{figure}
	    		\begin{center}
	    			\includegraphics[scale=0.1]{images/2icons/q.png}
	    		\end{center}
	    	\end{figure}
	    	\column{.2\textwidth}
	    	\begin{figure}
	    		\begin{center}
	    			\includegraphics[scale=0.1]{images/2icons/rq.png}
	    		\end{center}
	    	\end{figure}
	    \end{columns}
\end{frame}

\begin{frame}
    \frametitle{Selección de las Fuentes de Investigación (3/3)\\ Bibliotecas Digitales}
    \begin{table}
    	\begin{center}
    		\begin{tabular}{| p{5cm} | p{5cm} |}
    			\hline
    			\textbf {Fuentes Contempladas en un Inicio} &  \textbf{Fuentes Donde se Realizó la Búsqueda}\\
    			\hline
    			IEEE Xplore Digital Library & IEEE Xplore Digital Library\\
    			\hline
    			ACM Digital Library & ACM Digital Library\\
    			\hline
    			Web of Science & Web of Science\\
    			\hline
    			JSTOR & x\\
    			\hline
    			Scopus & Scopus\\
    			\hline
    			Springer Link & Springer Link\\
    			\hline
    			Elsevier Science Direct & Elsevier Science Direct\\
    			\hline
    			Google Scholar & x\\
    			\hline
    		\end{tabular}
    	\end{center}
    \end{table}
    \begin{columns}[c]
    	\column{.2\textwidth}
    	\begin{figure}
    		\begin{center}
    			\includegraphics[scale=1.2]{images/2icons/library.png}
    		\end{center}
    	\end{figure}
    	\column{.2\textwidth}
    	\begin{figure}
    		\begin{center}
    			\includegraphics[scale=0.2]{images/2icons/books.png}
    		\end{center}
    	\end{figure}
    \end{columns}
\end{frame}


\begin{frame}
    \frametitle{Procedimiento de Selección de Estudios\\ Criterios de Inclusión y Exclusión}
	
	\begin{table}
		\begin{center}
			\begin{tabular}{| p{5.5cm} | p{5.5cm} |}
				\hline
				\multicolumn{1}{|c|}{\textbf{Criterios de Inclusión}} & \multicolumn{1}{|c|}{\textbf{Criterios de Exclusión}} \\
				\hline
				Incluir artículos donde el título contengan dos palabras clave de nuestras cadenas de búsqueda principalmente ``Gamification''{ }& Excluir todos los artículos donde el título no contengan las palabras clave de la cadena de búsqueda.\\
				\hline
				Incluir artículos donde el resumen presente una relación con la aplicación de técnicas de  gamificación. & Excluir artículos donde el resumen no tenga relación con la aplicación de técnicas de gamificación.\\
				\hline
				Incluir artículos que estén escritos en Ingles y Español. & Excluir artículos que no estén en los idiomas del criterio de inclusión.\\ \hline
				Incluir artículos entre el rango de años: 2010 - 2015. & Excluir artículos publicados menores al año 2010.\\
				\hline
			\end{tabular}
		\end{center}
	\end{table}
\end{frame}

\begin{frame}
    \frametitle{Evaluación de la Calidad de Estudios\\ Criterios de inclusión y exclusión aplicados para la calidad del Estudio}
    .\begin{table}
    	\begin{center}
    		\begin{tabular}{| p{6cm} | p{4cm} |}
    			\hline
    			\multicolumn{1}{|c|}{\textbf{Criterios de Inclusión}} & \multicolumn{1}{|c|}{\textbf{Criterios de Exclusión}} \\
    			\hline
    			Incluir información sobre métodos de implementación para establecer estrategias de aprendizaje por medio de ``Gamification''.{ }& Excluir toda la información que no este relacionada con los criterios de inclusión definidos.\\
    			\hline
    			Incluir información que contenga experiencias del aprendizaje con ``Gamification''. &{ } \\
    			\hline
    			Incluir información donde se muestren estudios para el aprendizaje por medio de técnicas de e-learning.& { }\\ \hline
    		\end{tabular}
    	\end{center}
    \end{table}
\end{frame}

\begin{frame}
    \frametitle{Extracción y Monitoreo de la Información}
    \footnotesize
        	    \begin{table}
        	    	\begin{center}
        	    		\begin{tabular}{| l | p{7cm} |}
        	    			\hline
        	    			\multicolumn{2}{|c|}{\textbf{REGISTRO DE ARTÍCULO}} \\ \hline
							NÚMERO & Identificador Propio \\ \hline
        	    			TÍTULO & Titulo del Artículo \\ \hline
        	    			CATEGORÍA & Gamification, Education, E-Learning, Serious Games \\ \hline
        	    			AÑO DE PUBLICACIÓN & Año de Publicación del Artículo \\ \hline
        	    			\begin{tabular}[c]{@{}l@{}}PALABRAS CLAVE \\ (KEYWORDS)\end{tabular} & Lista de Palabras Clave del Artículo \\ \hline
        	    			ABSTRACT & Resumen del Artículo \\ \hline
        	    			\begin{tabular}[c]{@{}l@{}}NOTAS ACERCA\\ DEL ABSTRACT \end{tabular} & Ideas principales del resumen del Artículo \\ \hline
        	    			\multicolumn{2}{|c|}{\textbf{EXTRACCIÓN DE LA INFORMACIÓN}} \\ \hline                        
        	    			\begin{tabular}[c]{@{}l@{}}RESUMEN DE\\ SECCIONES\end{tabular} & Estructura del documento con ideas principales de cada sección.\\ \hline
        	    			\begin{tabular}[c]{@{}l@{}}TÉCNICAS, MÉTODOS,\\ APLICACIONES;\\ CREADAS\end{tabular} & Descripción de las Técnicas, Métodos o Aplicaciones para el uso de Gamificación.\\ \hline
        	    			NOTAS  ADICIONALES & Material Extra e ideas para comprender el estudio.\\ \hline
        	    		\end{tabular}
        	    	\end{center}
        	    \end{table}	
\end{frame}


\begin{frame}
	\frametitle{Resumen Selección Estudios Primarios}
	\begin{figure}
		\begin{center}
			\includegraphics[scale=0.35]{images/1document/results.png}
		\end{center}
	\end{figure}
\end{frame}

\begin{frame}
    \frametitle{Sístensis de la Información\\ Año de Publicación}
	\begin{figure}
		\begin{center}
			\includegraphics[scale=0.4]{images/1document/PrimariosAnio.png}
		\end{center}
	\end{figure}
\end{frame}

\begin{frame}
	\frametitle{Sístensis de la Información\\ Lugar de Publicación}
	\begin{figure}
		\begin{center}
			\includegraphics[scale=0.35]{images/1document/PrimariosLugar2.png}
		\end{center}
	\end{figure}
\end{frame}

\begin{frame}
	\frametitle{Sístensis de la Información\\ Tema Principal del Artículo}
	\begin{figure}
		\begin{center}
			\includegraphics[scale=0.55]{images/1document/theme.png}
		\end{center}
	\end{figure}
\end{frame}

%------------------------------------------------
\subsection{Reporte de Resultados} %
%------------------------------------------------

\begin{frame}
\Huge{\centerline{Reporte de Resultados}}
\end{frame}

\begin{frame}
    \frametitle{Métodos y Técnicas Aplicados}
    \begin{figure}
    	\begin{center}
    		\includegraphics[scale=0.4]{images/1document/comparative.png}
    	\end{center}
    \end{figure}
\end{frame}
%------------------------------------------------
\section{Conclusiones} %
%------------------------------------------------
\begin{frame}
    \frametitle{Conclusiones}
    Hasta este momento contra el objetivo general y los objetivos específicos no se han cumplido, porque el estado del arte aún no se encuentra definido. Sin embargo, actualmente se están analizando los estudios primarios para lograrlo.\\~\\
    Por otro lado, los estudios primarios, contienen un buen marco de referencia para establecer el estado del arte en el tema de Gamificación en la Educación.\\
    
\end{frame}
%------------------------------------------------

\begin{frame}
\frametitle{Referencias}
\tiny{
	\begin{itemize}
		\item Barata, G., Gama, S., Jorge, J., \& Gonçalves, D. (2013). Improving Participation and Learning with Gamification, 10–17. http://doi.org/10.1145/2583008.2583010
		\item Dunwell, I., Freitas, S. De, Petridis, P., Hendrix, M., Arnab, S., Lameras, P., \& Stewart, C. (2014). A Game-Based Learning Approach to Road Safety: The Code of Everand, 3389–3398.
		\item O'donovan, S., Gain, J., \& Marais, P. (2013). A case study in the gamification of a university-level games development course. Proceedings of the South African Institute for Computer Scientists and Information Technologists Conference on - SAICSIT '13, 242–251. http://doi.org/10.1145/2513456.2513469
			\item Pedro, L. Z., Lopes, A. M. Z., Prates, B. G., Vassileva, J., \& Isotani, S. (2015). Does Gamification Work for Boys and Girls? An Exploratory Study with a Virtual Learning Environment. ACM Symposium on Applied Computing, 214–219. http://doi.org/10.13140/RG.2.1.4783.5686
			\item Villagrasa, S., \& Duran, J. (2013). Gamification for learning 3D computer graphics arts. Proceedings of the First International Conference on Technological Ecosystem for Enhancing Multiculturality - TEEM ’13, 429–433. http://doi.org/10.1145/2536536.2536602
			\item Villagrasa, S., Fonseca, D., \& Durán, J. (2014). Teaching Case: Applying Gamification Techniques and Virtual Reality for Learning Building Engineering 3D Arts. Proceeding of the 2nd International Conference on Technological Ecosystems for Enhancing Multiculturality, 171–177.		
	\end{itemize}
}
\end{frame}
%------------------------------------------------

\begin{frame}
\Huge{\centerline{Gracias :)}}
\end{frame}

\begin{frame}
	\titlepage
	
\end{frame}


%----------------------------------------------------------------------------------------

\end{document} 