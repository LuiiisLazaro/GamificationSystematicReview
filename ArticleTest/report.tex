%%%%%%%%%%%%%%%%%%%%%%%%%%%%%%%%%%%%%%%%%
% Stylish Article
% LaTeX Template
% Version 2.1 (1/10/15)
%
% This template has been downloaded from:
% http://www.LaTeXTemplates.com
%
% Original author:
% Mathias Legrand (legrand.mathias@gmail.com) 
% With extensive modifications by:
% Vel (vel@latextemplates.com)
%
% License:
% CC BY-NC-SA 3.0 (http://creativecommons.org/licenses/by-nc-sa/3.0/)
%
%%%%%%%%%%%%%%%%%%%%%%%%%%%%%%%%%%%%%%%%%

%----------------------------------------------------------------------------------------
%	PACKAGES AND OTHER DOCUMENT CONFIGURATIONS
%----------------------------------------------------------------------------------------

\documentclass[fleqn,10pt]{SelfArx} % Document font size and equations flushed left

\usepackage[spanish]{babel} % Specify a different language here - english by default
\usepackage{multirow}

%----------------------------------------------------------------------------------------
%	COLUMNS
%----------------------------------------------------------------------------------------

\setlength{\columnsep}{0.55cm} % Distance between the two columns of text
\setlength{\fboxrule}{0.75pt} % Width of the border around the abstract

%----------------------------------------------------------------------------------------
%	COLORS
%----------------------------------------------------------------------------------------

\definecolor{color1}{RGB}{0,0,90} % Color of the article title and sections
\definecolor{color2}{RGB}{0,20,20} % Color of the boxes behind the abstract and headings

%----------------------------------------------------------------------------------------
%	HYPERLINKS
%----------------------------------------------------------------------------------------

\usepackage{hyperref} % Required for hyperlinks
\hypersetup{hidelinks,colorlinks,breaklinks=true,urlcolor=color2,citecolor=color1,linkcolor=color1,bookmarksopen=false,pdftitle={Title},pdfauthor={Author}}

%----------------------------------------------------------------------------------------
%	ARTICLE INFORMATION
%----------------------------------------------------------------------------------------

\JournalInfo{Estudio Guiado I, Vol. I, No. 1, 1-X, 2015} % Journal information
\Archive{Dra. Mirna Ariadna Muños Mata} % Additional notes (e.g. copyright, DOI, review/research article)

\PaperTitle{GAMIFICACIÓN EN PLATAFORMAS E-LEARNING} % Article title

\Authors{Luis Angel Hernández Lázaro\textsuperscript{1}*} % Authors
\affiliation{\textsuperscript{1}\textit{Maestría en Ingeniería de Software, Centro de Investigación en Matemáticas, Zacatecas, Zacatecas, México}} % Author affiliation
\affiliation{*\textbf{Autor Correspondiente}: luis.hernandez@cimat.mx} % Corresponding author

\Keywords{Gamification --- Education --- E-Learining --- Gamify --- Serious Games} % Keywords - if you don't want any simply remove all the text between the curly brackets
\newcommand{\keywordname}{Keywords} % Defines the keywords heading name

%----------------------------------------------------------------------------------------
%	ABSTRACT
%----------------------------------------------------------------------------------------

\Abstract{
as
}

%----------------------------------------------------------------------------------------

\begin{document}

\flushbottom % Makes all text pages the same height

\maketitle % Print the title and abstract box

\tableofcontents % Print the contents section

\thispagestyle{empty} % Removes page numbering from the first page

%----------------------------------------------------------------------------------------
%	ARTICLE CONTENTS
%----------------------------------------------------------------------------------------

%----------------------------------------------------------------------------------------
%	INTRODUCTION
%----------------------------------------------------------------------------------------

\section*{Introduction} % The \section*{} command stops section numbering

\addcontentsline{toc}{section}{Introduction} % Adds this section to the table of contents
CREAR INTRODUCCIÓN DEL TRABAJO \\
ALENTAR A LA LECTURA\\
:D\\


%----------------------------------------------------------------------------------------
%	BACKGROUND
%----------------------------------------------------------------------------------------

\section{Background}
En la actualidad existen plataformas de educación online, conocidas como E-Learning, que promueven el aprendizaje por mdeio de cursos y asesoría virtual, apoyados por las características que ofrece la tecnología de hoy en día, sin embargo algunas plataformas no hacen uso de técnicas de Gamificación para incentivar a sus usuarios. \textbf{Significando lo cual} solo existe el contenido de manera similar a la forma tradicional de aprendizaje, es decir, aula - alumno - profesor. \\
Por lo tanto el desarrollo de este trabajo permitirá analizar cuales son las técnicas o métodos aplicados en el área de la educación para generar un grupo de técnicas que sean factible de aplicar en plataformas e-learning.\\  \\

AGREGAR TÉRMINOS Y CONCEPTOS BÁSICOS ENTRE LAS PALABRAS CLAVES OBTENIDOS CON LAS REFERENCIAS DE LOS ESTUDIOS PRIMARIOS.\\ \\


El Objetivo General de estudio es:
\begin{enumerate}
	\item Definir el estado del actual del uso de técnicas de Gamificación en la educación tradicional (aula - alumno - profesor), para descubrir la factibilidad de implementar de técnicas de Gamificación en plataformas E-Learning.
\end{enumerate}
Los objetivos Específicos del estudio son:
\begin{enumerate}
	\item Definir el estado del arte de la Gamificación en E-Learning.
	\item Revisar el uso de estrategias o técnicas de Gamificación en el área de la educación.
	\item Comparar la aplicación de gamificación en educación.
\end{enumerate}

%----------------------------------------------------------------------------------------
%	SYSTEMATIC REVIEW STEPS 
%----------------------------------------------------------------------------------------

\section{El Proceso de la Revisión Sistemática}
Una revisión sistemática se basa en el objetivo para recolectar estudios de un tema en especial a investigar, originando la búsqueda de nuevos conocimientos para nuestra formación académica. ``Una Revisión sistemática de la literatura permite identificar, evaluar, interpretar y sintetizar todas las investigaciones existentes y relevantes en un tema de interés en particular" \textbf{PONER CITA DEL ARTICULO}.\\
La estimulación para crear esta revisión sistemática es mostrar cuales son los resultados de las técnicas de gamificación aplicadas en la educación.\\
Algunas de las características que diferencian una revisión sistemática de una revisión de la literatura convencional son\textbf{PONER CITA DEL ARTÍCULO}:
\begin{enumerate}
	\item La revisión sistemática comienza por definir un protocolo de revisión que especifica la(s) pregunta(s) de investigación abordados y los métodos que serán usados para realizar la revisión.
	\item Las revisiones sistemáticas son basadas definiendo una estrategia de búsqueda que tiene como objetivo detectar la mayor cantidad de literatura relevante posible.
	\item Las revisiones sistemáticas documentan sus estrategias de búsqueda para que los lectores puedan evaluar su rigor y exahustividad y la repetitividad del proceso.
	\item Las revisiones sistemáticas requieren criterios de inclusión y criterios de exclusión explícitos para evaluar cada estudio primario potencial.
	\item Las revisiones sistemáticas especifican la información para obtener cada uno de los estudios primarios incluyendo los criterios de calidad para evaluar cada estudio primario.
\end{enumerate}
En la tabla \ref{table:stepsSystematicReview} se muestran las etapas principales con sus respectivos pasos, que confirman una revisión sistemática.
\begin{table}
    \begin{center}
        \caption{Pasos de la Revisión Sistemática}
        \label{table:stepsSystematicReview}
        \begin{tabular}{| p{2.3cm} | p{5.3cm} |}
            \toprule
            \hline 
            \multicolumn{1}{|c|}{\textbf{Etapa}} & \multicolumn{1}{|c|}{\textbf{Paso}}\\
            \hline
            \multicolumn{1}{|c|}
                {\multirow{4}{*}
                    {\begin{tabular}[c]{@{}c@{}}
                        Planificación de\\ la Revisión
                    \end{tabular}}
                } & Identificación de la necesidad para realizar la revisión.\\ \cline{2-2} 
                \multicolumn{1}{|c|}{} & Especificación de la(s) pregunta(s) de investigación.\\ \cline{2-2} 
                \multicolumn{1}{|c|}{} & Desarrollo del protocolo de revisión.\\ \cline{2-2} 
                \multicolumn{1}{|c|}{} & Evaluación del protocolo de revisión.\\ 
            \hline
            \multicolumn{1}{|c|}
                {\multirow{5}{*}
                    {\begin{tabular}[c]{@{}c@{}}
                        Ejecución de\\ la Revisión
                    \end{tabular}}
                } & Identificación de la investigación.\\ \cline{2-2} 
                \multicolumn{1}{|c|}{} & Selección de estudios primarios.\\ \cline{2-2} 
                \multicolumn{1}{|c|}{} & Evaluación de la calidad de los estudios.\\ \cline{2-2} 
                \multicolumn{1}{|c|}{} & Extracción y monitoreo de datos.\\ \cline{2-2} 
                \multicolumn{1}{|c|}{} & Sintetizar los datos.\\ 
            \hline
            \multicolumn{1}{|c|}
                {\multirow{3}{*}
                    {\begin{tabular}[c]{@{}c@{}}
                        Ejecución de\\ la Revisión
                    \end{tabular}}
                } & Especificación de los mecanismos de difusión.\\ \cline{2-2} 
                \multicolumn{1}{|c|}{} & Formateo del informe principal.\\ \cline{2-2} 
                \multicolumn{1}{|c|}{} & Evaluación del reporte.\\ 
            \hline
        \end{tabular}
    \end{center}
\end{table}
En las siguientes partes del documento se muestra el desarrollo para cada una de las etapas realizadas en está revisión sistemática.

%----------------------------------------------------------------------------------------
%	PLAINNING
%----------------------------------------------------------------------------------------

	\subsection{Planeación de la Revisión Sistemática}
	Al realizar nuestra revisión sistemática es importante la etapa de la planeación, para establecer las actividades a ejecutar, de esta forma definimos nuestro marco de trabajo durante la ejecución de la revisión sistemática, es importante realizar la planeación de forma correcta, así como la identificación de la necesidad de la revisión para evitar caer en problemas durante su ejecución.\\
	En esta primera etapa, la actividad más importante a tener en cuenta es la definición de la(s) pregunta(s) de investigación, puesto que son el punto clave para el desarrollo de nuestra investigación.
	
%----------------------------------------------------------------------------------------
%	NEED
%----------------------------------------------------------------------------------------

	\subsubsection{Identificación de la Necesidad de la Revisión Sistemática}
PRIMERA IDEA:\\	
	El término Gamificación actualmente esta tomando fuerza, teniendo en cuenta la aplicación más allá de los juegos que comúnmente conocemos. 
	\\HABLAR MÁS DEL PORQUE DE LA NECESIDAD DEL TRABAJO PARA ESTA REVISIÓN SISTEMÁTICA.
	\\
	MÁS IDEAS\\

%----------------------------------------------------------------------------------------
%	RESEARCH QUESTIONS
%----------------------------------------------------------------------------------------

	\subsubsection{Especificación de las Preguntas de Investigación}
    La revisión sistemática es desarrollada para identificar y conocer el estado del arte referente a la gamificación en el área de la educación. Sin embargo también se desea conocer como ha sido aplicada la Gamificación en diferentes áreas para promover el aprendizaje, con el objetivo de identificar cuál ha sido su efectividad en la implementación de técnicas de Gamificación. En la Tabla \ref{table:researchQuestions} se muestran las preguntas de investigación formuladas para esta investigación, las cuales serán contestadas al realizar esta revisión sistemática.\\

\begin{table}
    \begin{center}
        \caption{Preguntas de Investigación}
        \label{table:researchQuestions}
        \begin{tabular}{| p{0.6cm} | p{3.5cm} | p{3.5cm} |}
            \toprule
            \hline
            \multicolumn{1}{|c|}{\textbf{RQ\#}} & \multicolumn{1}{|c|}{\textbf{Pregunta}}  & \multicolumn{1}{|c|}{\textbf{Objetivo}} \\
            \hline
            RQ1 & ¿Qué elementos de gamificación son usados en la educación? & Identificar los elementos de Gamificación existentes, en particular en la educación. \\
            \hline
            RQ2 & ¿Qué tipo de estudios de investigación aplican gamificación en la educación? & Descubrir los trabajos que han sido desarrollados para aplicar la gamificación en la educación. \\
            \hline
            RQ3 & ¿Qué técnicas de gamificación han sido más efectivas? & De las técnicas existentes cuales han tenido resultados positivos en la educación y porque. \\
            \hline
            RQ4 & ¿La gamificación en e-learning promueve el aprendizaje? & De de las técnicas de gamificación cuales se han aplicado para el área de la educación.\\
            \hline
            RQ5 & ¿En qué contextos ha sido investigada la gamificación? & Que experiencias se han tenido en el uso de gamificación.\\
            \hline
        \end{tabular}
    \end{center}
\end{table}

    \subsubsection{Desarrollo del Protocolo de Revisión}
    EXPLICAR LA MANERA PARA REALIZAR LA REVISIÓN SISTEMÁTICA\\
    DEFINIR EL TEMA\\
    LAS BIBLIOTECAS\\
    CADENAS DE BÚSQUEDA\\

    \subsubsection{Evaluación del Protocolo de Revisión}
    EXPLICAR QUE SE REALIZÓ CON LOS CORREOS ENVIADOS AL ASESOR.\\
    
%----------------------------------------------------------------------------------------
%	SYSTEMATIC REVIEW EXECUTE
%----------------------------------------------------------------------------------------


\subsection{Ejecución de la Revisión}
    INTRODUCCIÓN A LA REVISIÓN SISTEMÁTICA\\    
    PRIMERA IDEA: Ahora que hemos establecido cuales son nuestras preguntas de investigación y el protocolo de investigación, comenzamos la etapa de ejecución. Durante la ejecución de la revisión sistemática debemos respetar el procedimiento establecido en el protocolo.

    \subsubsection{Selección de las Fuentes / Identificación de la Investigación}
    La búsqueda para encontrar los estudios más relevantes se realizó en bibliotecas digitales haciendo uso de las cadenas de búsqueda con las siguientes palabras clave:
    \begin{enumerate}
        \item Gamification
        \item Gamify
        \item Serious Games
        \item Education
        \item E-Learning
    \end{enumerate}
    En la Tabla \ref{table:SearchString} se muestran las cadenas de búsqueda aplicadas a las bibliotecas digitales, como se puede notar las cadenas de búsqueda están formadas por las palabras claves de nuestra investigaciones y se combinaron mediante el uso de los conectores lógicos ``OR"{ }y ``AND". \\

\begin{table}
    \begin{center}
        \caption{Cadenas de Búsqueda}
        \label{table:SearchString}
        \begin{tabular}{| p{0.6cm} | p{7cm} |}
            \toprule
            \hline
            \multicolumn{1}{|c|}{\textbf{Núm}} & \multicolumn{1}{|c|}{\textbf{Cadena de Búsqueda}} \\
            \hline
            ss1 & ( ``Gamification"{ }AND (``Education"{ }OR ``E-Learning"{ }) )\\
            \hline
            ss2 & ( ``Serious Games"{ }AND ``Gamification"{ }AND (``Education"{ }OR ``E-Learning"{ }) )\\
            \hline            
            ss3 & ( ``Gamify"{ }AND (``Education"{ }OR ``E-Learning"{ }) )\\
            \hline
        \end{tabular}
    \end{center}
\end{table}

Mencionar el inicio de bibliotecas digitales a tener en cuenta para aplicar las cadenas de búsqueda.

\begin{table}
    \begin{center}
        \caption{Fuentes de Búsqueda}
        \label{table:libraries}
        \begin{tabular}{| p{3.8cm} | p{3.8cm} |}
            \toprule
            \hline
            \textbf {Fuentes Contempladasen un Inicio} &  \textbf{Fuentes donde se realizó la búsqueda}\\
            \hline
            IEEE Xplore Digital Library & IEEE Xplore Digital Library\\
            \hline
            ACM Digital Library & ACM Digital Library\\
            \hline
            Web of Science & Web of Science\\
            \hline
            JSTOR & x\\
            \hline
            Scopus & Scopus\\
            \hline
            Springer Link & Springer Link\\
            \hline
            Elsevier Science Direct & Elsevier Science Direct\\
            \hline
            Google Scholar & x\\
            \hline
        \end{tabular}
    \end{center}
\end{table}

%----------------------------------------------------------------------------------------
%	SELECT LIBRARYS
%----------------------------------------------------------------------------------------

	\subsubsection{Procedimiento de Selección de los Estudios Primarios}
	En esta actividad, se describe la manera en la que fueron seleccionados los estudios, después de realizar la búsqueda con nuestras cadenas de búsqueda definidas en la tabla \ref{table:SearchString} aplicadas en las bibliotecas digitales tabla \ref{table:libraries}. Al ejecutar la búsqueda en cada una de las bibliotecas seleccionadas, debemos aplicar nuestros criterios de inclusión y exclusión para destacar los estudios candidatos a convertirse a estudios primarios.\\
	Los criterios de inclusión y exclusión permiten acortar el número de resultados y enfocarnos en estudios de nuestro interés, relevantes para nuestra investigación y con un alto impacto de conocimiento. En esta investigación los criterios de inclusión y exclusión se muestran en la tabla \ref{table:criterios}.
	
\begin{table}
    \begin{center}
        \caption{Criterios de inclusión y exclusión}
        \label{table:criterios}
        \begin{tabular}{| p{3.8cm} | p{3.8cm} |}
            \toprule
            \hline
            \multicolumn{1}{|c|}{\textbf{Criterios de Inclusión}} & \multicolumn{1}{|c|}{\textbf{Criterios de Exclusión}} \\
            \hline
            Incluir artículos donde el título contengan dos términos de nuestras cadenas de búsqueda principalmente ``Gamification"{ }& Excluir todos los artículos que correspondan con los criterios de inclusión definidos.\\
            \hline
            Incluir artículos que contengan dentro de sus palabras clave por lo menos dos términos de nuestras cadenas de búsqueda. & Excluir todos los artículos duplicados para eliminar la redundancia de información.\\
            \hline
            Incluir artículos donde el resumen presente una relación con la aplicación de técnicas de gamificación. & Excluir artículos que se encuentren en idiomas diferentes al español o ingles.\\
            \hline
            . & Excluir artículos que contengan más de 10 páginas, debido a la limitación de tiempo para realizar la RS, con excepción de los artículos que sean de suma importancia para la investigación.\\
            \hline
        \end{tabular}
    \end{center}
\end{table}

    Durante la ejecución de las cadenas de búsqueda, si el artículo encontrado en los resultados solo contenía un término de búsqueda, se procedió a revisar el título para ver la relación con el tema de nuestra investigación, si existían dudas acerca del título del artículo, se prosiguió a la lectura del abstract (resumen) para determinar si el artículo calificaba como estudio relevante. 
    \\
    A partir de los resultados de nuestras cadenas de búsqueda se muestra la distribución de estudios relevantes en la tabla \ref{table:ResultsSearchString}. La aplicación de las cadenas de búsqueda se realizó de la siguiente manera: se aplicó la cadena ss1, posteriormente la cadena ss2 y por último la cadena ss3; la búsqueda se realizó a las bibliotecas en el orden descrito en \ref{orderLibraries}.
    
    \begin{enumerate}
        \label{orderLibraries}
            \item IEEE Xplore Digital Library.
            \item ACM Digital Library.
            \item Web of Science.
            \item Scopus.
            \item Springer Link.
            \item Elsevier Science Direct.
    \end{enumerate}    

    \begin{table}
    \begin{center}
        \caption{Resultados de Cadenas de Búsqueda}
        \label{table:ResultsSearchString}
        \begin{tabular}{| p{2.5cm} | p{.6cm} | p{.6cm} | p{.8cm} | p{1.5cm} |}
            \toprule
            \hline
            \textbf{Fuente /Cadena} & \textbf{ss1} & \textbf{ss2} & \textbf{ss3} & \textbf{Total por Biblioteca Digital}\\ \hline
            ACM Digital Library & 123 & 93 & 45 & 261\\ \hline
            Elsevier Science Direct & 301 & 93 & 27 & 421\\ \hline
            IEEE Xplore Digital Library & 102 & 4 & 77151\textsuperscript{*} & 106\\ \hline
            Scopus & 359 & 49 & 11 & 419\\ \hline
            Springer Link & 736 & 284 & 64 &  1084\\ \hline
            Web of Science & 139 & 18 & 0 & 157\\ \hline
            \textbf{Total por Cadena de Búsqueda} & \textbf{1760} & \textbf{541} & \textbf{147} & \textbf{2448}\\ \hline
        \end{tabular}
    \end{center}
    \end{table}
    Para cada una de las bibliotecas digitales se analizaron los resultados con base en los criterios de inclusión (tabla \ref{table:criterios}) para obtener los estudios relevantes del tema, con la finalidad de pulir la búsqueda para la selección de estudios primarios. Los resultados de las cadenas de búsqueda con los criterios de inclusión y exclusión se pueden observar en la tabla \ref{table:ResultsSearchStringOutStandingStudies}.

    \begin{table}
    \begin{center}
        \caption{Resultados de Cadenas de Búsqueda, Estudios Relevantes}
        \label{table:ResultsSearchStringOutStandingStudies}
        \begin{tabular}{| p{2.5cm} | p{.6cm} | p{.6cm} | p{.8cm} | p{1.5cm} |}
            \toprule
            \hline
            \textbf{Fuente /Cadena} & \textbf{ss1} & \textbf{ss2} & \textbf{ss3} & \textbf{Total por Biblioteca Digital}\\ \hline
            ACM Digital Library & 13 & 4 & 3 & 20 \\ \hline
            Elsevier Science Direct & 7 & 1 & 0 & 8\\ \hline
            IEEE Xplore Digital Library & 17 & 1 & 77151\textsuperscript{*} & 18\\ \hline
            Scopus & 0 & 0 & 2 & 2\\ \hline
            Springer Link & 16 & 2 & 1 &  19\\ \hline
            Web of Science & 9 & 0 & 0 & 9 \\ \hline
            \textbf{Total por Cadena de Búsqueda} & \textbf{62} & \textbf{8} & \textbf{6} & \textbf{76}\\ \hline
        \end{tabular}
    \end{center}
    \end{table}
    
    Durante la selección de estudios relevantes, existió la repetición de artículos en diferentes bibliotecas digitales, la mayoría de artículos repetidos se dio entre: 
%----------------------------------------------------------------------------------------
%	PRIMARY STUDIES
%----------------------------------------------------------------------------------------

	\subsubsection{Evaluación de la Calidad de los Estudios}
    w
%----------------------------------------------------------------------------------------
%	INFORMATION EXTRACTION
%----------------------------------------------------------------------------------------

	\subsubsection{Extracción y Monitoreo de la Información}
    w
%----------------------------------------------------------------------------------------
%	RESUMEN
%----------------------------------------------------------------------------------------
	
	\subsubsection{Síntesis de la Información}
    w
%----------------------------------------------------------------------------------------
%	RESULT
%----------------------------------------------------------------------------------------

\subsection{Reporte de Resultados}
    w
    \subsubsection{Revisiones Literarias}
	w
	\subsubsection{Métodos Existentes}
	w
	\subsubsection{Evaluación del Reporte}
    w
%----------------------------------------------------------------------------------------
%	CONCLUSIONS
%----------------------------------------------------------------------------------------
	
\section{Conclusiones}
    w
%------------------------------------------------
\phantomsection
\section*{Acknowledgments} % The \section*{} command stops section numbering

\addcontentsline{toc}{section}{Acknowledgments} % Adds this section to the table of contents

%----------------------------------------------------------------------------------------
%	REFERENCE LIST
%----------------------------------------------------------------------------------------
\phantomsection
\bibliographystyle{unsrt}
\bibliography{sample}

%----------------------------------------------------------------------------------------

\end{document}
